\usepackage{amsmath}
\usepackage{amsfonts}
\usepackage{amssymb}
\usepackage{bm} % bold math
\usepackage{commath2,commath2-additions}
% \usepackage{cancel} % strikethrough/cancel equations
% \usepackage{authblk} % author/affiliation..?
\usepackage{import}
\usepackage{graphicx}
% \usepackage[font=small,labelfont=bf,width=.9\textwidth]{caption}
\usepackage[textwidth=2.5cm]{todonotes}
\newcommand{\inlinetodo}[1]{\todo[inline]{#1}}
\usepackage[section]{placeins} % forces floats (images) to stay within their section (stops results images from flowing into conclusion)
\usepackage{siunitx}
\sisetup{%
    load-configurations = abbreviations, %
    per-mode = symbol, %
    % range-phrase = -- %
}%
\usepackage{booktabs} % better tables, \hline --> \toprule, \midrule, \bottomrule
\usepackage{multirow}
\usepackage{printlen} % to print lengths like \textwidth, \textheight using \printlength{\textwidth}
\uselengthunit{mm}
% \usepackage{subcaption}
\usepackage{pbox}

% highlighting
\usepackage{color} % Must be loaded before soul, for highligting
\usepackage{soul}
\usepackage{xcolor}
\newcommand{\redtext}[1]{\textcolor{red}{#1}}
% \renewcommand{\hl}[2][yellow]{%
%     \sethlcolor{#1}%
%     \hl{#2}%
% }%
\DeclareRobustCommand{\hlg}[1]{%
    \sethlcolor{green}%
    \hl{#1}%
}%
\DeclareRobustCommand{\hly}[1]{%
    \sethlcolor{yellow}%
    \hl{#1}%
}%
% \newcommand{\greenbox}[1]{\colorbox{green}{#1}}
% \newcommand{\yellowbox}[1]{\colorbox{yellow}{#1}}
\newcommand{\greenbox}[1]{\hlg{#1}}
\newcommand{\yellowbox}[1]{\hly{#1}}

\usepackage{hyperref}
\usepackage{cleveref}

% Custom bracket sizes \vast and \Vast
\makeatletter
\newcommand{\vast}{\bBigg@{3}}
\newcommand{\Vast}{\bBigg@{4}}
\makeatother

% math symbols
\renewcommand{\textmu}{\textmugreek}
\newcommand{\bvec}[1]{\bm{#1}}
\renewcommand{\vec}[1]{\bvec{#1}}
\newcommand{\vect}[1]{\bm{#1}}
\newcommand{\upmu}{\textup{\textmu}}
\newcommand*{\epsO}{\epsilon_0}
\let\div\olddiv
\DeclareMathOperator{\div}{div}
\let\oldtimes\times
\let\times\cdot
% \DeclareMathOperator{\abs}{abs}
\newcommand{\overbar}[1]{\mkern 1.5mu\overline{\mkern-1.5mu#1\mkern-1.5mu}\mkern 1.5mu}

% \newcommand{\Reyn}{\operatorname{\mathit{R\kern-.04em e}}} % Reynolds number
% \newcommand{\Nuss}{\operatorname{\mathit{N\kern-.09em u}}} % Nusselt number
% \newcommand{\Pran}{\operatorname{\mathit{P\kern-.03em r}}} % Prandtl number

\newcommand{\Reyn}{\operatorname{R\kern-.04em e}} % Reynolds number
\newcommand{\Nuss}{\operatorname{N\kern-.09em u}} % Nusselt number
\newcommand{\Pran}{\operatorname{P\kern-.03em r}} % Prandtl number