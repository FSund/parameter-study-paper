Natural gas exported from Norway to Europe accounts for around 25 percent of the yearly gas consumption in the European Union%\todo{source?}. %
. The gas is transported from Norway through pipelines that are up to 1166 km long. %
To ensure that the pipelines stay within their operating limits, to monitor the pipelines for leaks, and to track changes in gas quality, %
it is important to know the state of the gas in the pipelines. %
But measurements of the state of the gas are only available at the inlet and outlet, which means that numerical models are necessary to know the state of the gas between the endpoints. 

Simulating compressible gas flow is a highly complex issue, so to reduce the problem to a tractable one, several empirical relations and correlations like the Colebrook-White \cite{Colebrook1939Turbulent} equation and the Dittus-Boelter equation \cite{Winterton1998Where,Dittus1985Heat} are typically used to model different aspects of the system. When doing this, errors are introduced into the simulations, the total effect of which can be hard to calculate \emph{a priori}, and which will depend on the state and system being simulated. 

The objective of this study is to investigate which parameters and correlations in the gas models that have greatest impact on the modelled results, especially during transient conditions, to know where to apply effort when trying to improve the models. A similar study limited to steady state models was done by Langelandsvik \cite{Langelandsvik2008Modeling}, and some work using transient models was by Helgaker \cite{Helgaker2013Modeling}. The present work deals with transient one-dimensional non-isothermal models for compressible natural gas mixtures, and a simplified pipeline is modelled using synthetic but representative flow transients as boundary conditions.

This article is structured as follows: The theoretical foundation and underlying equations are presented in \cref{theory} %
%The numerical scheme is presented in \cref{num} 
followed by the presentation of the studied pipeline system in \cref{simulations}. Results are presented and discussed in \cref{results} while concluding remarks are drawn in \cref{conclusion}.