Which parameters to include in the sensitivity study were determined by looking at which variables appear in the governing equations (\cref{eq:continuityEquation,eq:momentumEquation,eq:energyEquation}), in addition to other correlations that are used in the simulations. The following 9 parameters are included in the study:
\begin{itemize}
    \item the Colebrook-White correlation for the friction factor $f$, \cref{eq:colebrookWhite}
    
    \item the compressibility factor $Z$ and three derivatives: $\eval{\pd{Z}{T}}_p$, $\eval{\pd{Z}{p}}_T$ and $\eval{\pd{Z}{T}}_\rho$, which are all calculated from the equation of state
    
    \item Nusselt number relations (the Dittus-Boelter equation) for inner and outer film heat transfer coefficients ($h_\text{inner}$ and $h_\text{outer}$), which go into the calculation of the heat transfer between the gas and the surroundings $\Omega$
    
    \item the correlation for heat capacity of the gas at constant volume %
    %(derived from the equation of state) 
    $c_v$
    
    \item the Lee-Gonzales-Eakin correlation for viscosity of the gas $\mu$ \cite{Lee1966Viscosity}, which mainly enters the simulations via the Reynolds number, $\Reyn = \frac{\rho u D}{\mu}$
 \end{itemize}

% To decide which parameters and relations to include in the sensitivity study we first looked to the governing equations \cref{eq:continuityEquation,eq:momentumEquation,eq:energyEquation}. Obvious choices are the compressibility factor $Z$ and derivatives, the friction factor $f$, the gas heat capacity $c_v$, and the heat transfer $\Omega$, which all appear directly in the partial differential equations. We find that the inner and outer heat transfer coefficients ($h_\text{inner}$ and $h_\text{outer}$) are the most relevant parameters to use to investigate the sensitivtity to changes in the heat transfer. We also include the correlation for viscosity (\redtext{LGE}).

% To investigate the sensitivity of the model a \emph{base case} was first established using standard model parameters. The simulation was then repeated with one parameter multiplied by a factor of 1.2, and the corresponding changes in the modelled flow, pressure, and temperature were recorded. To quantify the sensitivity for changes in the different parameters, both the maximum differences and the average differences from the base case was calculated. This was calculated both for the whole simulation period of 4 days, and for selected points of interest along the pipeline.

To investigate the sensitivity of the model a \emph{base case} was first established using standard model parameters and the boundary conditions described in \cref{subsec:pipeline}. The inlet flow rate transient was initiated at around 2 hours, and the pipeline was simulated for 104 hours after the transient.
%
The simulation was then repeated several times, with a different parameter modified by a constant factor of 1.2 each time. The corresponding response in the modelled flow, pressure, and temperature were recorded for each case. 

% To quantify the sensitivity for changes in the different parameters, both the maximum differences and the average differences from the base case was calculated, for the whole simulation period. 