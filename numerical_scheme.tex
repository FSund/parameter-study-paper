\subsection{Governing equations}
\label{subsec:governingNumerical}
To solve the non-linear partial differential equations for the three \emph{state variables} mass flow $\dot m$, pressure $p$, and temperature $T$, \crefrange{eq:continuityEquation}{eq:energyEquation} are first discretized using a scheme similar to the BTCS (backward time, centered space) finite difference scheme, using cell averages \cite{Kiuchi1993Implicit,Abbaspour2004Dynamic}. The pipeline is divided into $N$ grid points, and the different variables are approximated at each section between the grid points by
\begin{align}
    y \approx \frac{y^{n+1}_{i+1} + y^{n+1}_{i}}{2}
, \label{eq:variablesScheme}
\end{align}
where $y$ represents a general variable, superscripts $n$ and $n+1$ denote time level, and subscripts $i$ and $i+1$ denote grid points. Time derivatives are approximated by
\begin{align}
    % \pdt{y} =
    % \eval{\pdt{y}}_{\substack{x_I\\t_{n+1}}} =
    % y_t\del{x_I, t_{n+1}} =
    % \pdt{y}\del{x_I, t_{n+1}} =
    \pdt{y} \approx 
    % \pdt{y}\del{t + \Delta t} =
    \frac{y^{n+1}_{i+1} + y^{n+1}_{i} - \del{y^{n}_{i+1} + y^{n}_{i}}}{2\Delta t}
, \label{eq:timeScheme}
\end{align}
and spatial derivatives by
\begin{align}
    \pdx{y} \approx
    \frac{y^{n+1}_{i+1} - y^{n}_{i}}{\Delta x}
. \label{eq:spatialScheme}
\end{align}
This scheme is first order accurate in time, and second order accurate in space, \hlbr{and have been shown to give accurate results for pipelines and boundary conditions comparable to the ones used in the present study \cite{Abbaspour2008Nonisothermal,Helgaker2014Validation}. Central difference schemes are known to be very prone to oscillations, but this problem is avoided by choosing an appropriate time step and grid spacing, and by using smooth transients, avoiding discontinous changes in boundary conditions \cite{Abbaspour2008Nonisothermal}. A different approach could be using an upwind (backward difference) scheme for the spatial derivatives, but upwind schemes do not take into account acoustic information traveling from points which are downstream \cite{Helgaker2014Transient}, so central differences are preferred.}

When replacing \cref{eq:variablesScheme,eq:timeScheme,eq:spatialScheme} in \cref{eq:continuityEquation,eq:momentumEquation,eq:energyEquation} non-linear equations in $\dot m$, $p$ and $T$ are aquired. These equations are linearized by ``lagging'' behind parts of the non-linear terms \cite{thomas1998numerical}
\begin{align}
    y^{n+1} \rightarrow y^{n}
. \label{eq:lagTerms}
\end{align}
The result is a set of linear equations, with three equations for each pipe section, and $N-1$ total pipe sections, giving a total of $3(N-1)$ equations. The number of unknowns at time level $n+1$ is $3N$ ($N$ for each state variable), so three boundary conditions are needed. Here the inlet mass flow $\dot m_{1}$, outlet pressure $p_{N}$, and inlet temperature $T_{1}$ are chosen.
%  equations for .
%  which are solved using matrix inversion and the Jacobi iterative method. 
The linear equations with boundary conditions are written on matrix form
% \begingroup
% \renewcommand*{\arraystretch}{1.5}
% \begin{align}
%     \vec x = 
%     \begin{bmatrix}
%         \dot m^{n+1}_{i} \\
%         \dot m^{n+1}_{i+1} \\
%         \vdots \\
%         \dot m^{n+1}_{n-1} \\
%         \dot m^{n+1}_{n} \\
%         \dot p^{n+1}_{i} \\
%         \dot p^{n+1}_{i+1} \\
%         \vdots \\
%         \dot p^{n+1}_{n-1} \\
%         \dot p^{n+1}_{n} \\
%         \dot T^{n+1}_{i} \\
%         \dot T^{n+1}_{i+1} \\
%         \vdots \\
%         \dot T^{n+1}_{n-1} \\
%         \dot T^{n+1}_{n}
%     \end{bmatrix}
% \end{align}
% \endgroup

% \begingroup
% \renewcommand*{\arraystretch}{1.5}
% \begin{align}
%     \vec x = 
%     \begin{bmatrix}
%         \dot m^{n+1}_2 \\
%         \dot m^{n+1}_3 \\
%         \dot m^{n+1}_4 \\
%         \dot p^{n+1}_1 \\
%         \dot p^{n+1}_2 \\
%         \dot p^{n+1}_3 \\
%         \dot T^{n+1}_2 \\
%         \dot T^{n+1}_3 \\
%         \dot T^{n+1}_4
%     \end{bmatrix}
% \end{align}
% \endgroup

\begin{align}
    \vec A \vec x = \vec b
, \label{eq:matrixEquation}
\end{align}
where the vector $\vec x$ has length $3(N-1)$ and contains the unknowns
\begin{align}
    \vec x = 
    % \begin{bmatrix}
    %     \dot m^{n+1}_2 & 
    %     \dots &
    %     \dot m^{n+1}_N &
    %     \dot p^{n+1}_1 &
    %     \dots &
    %     \dot p^{n+1}_{N-1} &
    %     \dot T^{n+1}_2 &
    %     \dots &
    %     \dot T^{n+1}_N
    % \end{bmatrix}^{-1}
    \begin{bmatrix}
        \dot m^{n+1}_2, \dots, \dot m^{n+1}_N,
        p^{n+1}_1, \dots, p^{n+1}_{N-1}, 
        T^{n+1}_2, \dots, T^{n+1}_N
    \end{bmatrix}^{-1}
,
\end{align}
the matrix $\vec A$ has shape $3(N-1)\oldtimes 3(N-1)$ and contains the coefficients in front of the unknowns, and the vector $\vec b$ contains the known terms including the boundary conditions. \Cref{eq:matrixEquation} is solved using matrix inversion and the Jacobi iterative method \cite{Ferziger2002Computational}. This entails finding $\vec x$ using matrix inversion
\begin{align}
    \vec x = \vec A^{-1} \vec b
,
\end{align}
where the inverse $\vec A^{-1}$ is found using a linear algebra library. In the Jacobi iterative method the unknows at time level $n+1$ are given the values from $\vec x$, and terms like the friction factor, compressibility factor etc., are updated using the new mass flow, pressure and temperature. This gives a new set of coefficients $\vec A$ and known terms $\vec b$, and the procedure is repeated until the unknowns converge.